\chapter{Основная часть}

\section{Формальная постановка задачи}

Для построения поискового индекса в реляционной базе данных на основе нейронной
сети требуется:

\begin{itemize}
    \item исходный набор ключей в качестве входных данных;
    \item набор указателей на данные, соответсвующие ключам;
    \item правила их предварительной обработки, требующейся для обучения модели;
    \item модель нейронной сети в качестве основы будущего индекса;
    \item алгоритм обучения нейронной сети, результатом работы которого является
        обученная модуль, представляющая собой индекс.
\end{itemize}

Формально данная задача может быть описана с помощью IDEF0-диаграммы нулевого
уровня, представленной на рисунке~\ref{img:idef0-A0}.

\imgs{idef0-A0}{h!}{1}{Постановка задачи}

\section{Ограничения на входные и выходные данные}

\section{Основные этапы метода}

\section{Структура программного обеспечения}

\section{Выбор средств программной реализации}

\section{Ключевые моменты реализации}

\section{Взаимодействие с программным обеспечением}
