\chapter*{ВВЕДЕНИЕ}
\addcontentsline{toc}{chapter}{ВВЕДЕНИЕ}

Во время выполнения выпускной квалификационной работы был разработан метод
построения поисковых индексов в реляционной базе данных на основе глубоких
нейронных сетей.

Целью данной работы является разработка программного обеспечения,
демонстрирующего практическую осуществимость спроектированного в ходе выполнения
выпускной квалификационной работы метода.

Для достижения поставленной цели требуется решить следующие задачи:
\begin{itemize}
    \item описать формальную постановку задачи построения индекса в реляционной
        базе данных на основе глубоких нейронных сетей;
    \item описать ограничения, накладываемые на входные и выходные данные;
    \item отразить основные этапы метода в виде IDEF0-диаграмм;
    \item описать структуру программного обеспечения;
    \item описать выбор средств программной реализации;
    \item разработать программное обеспечение, реализующее метод построения
        индекса.
\end{itemize}
