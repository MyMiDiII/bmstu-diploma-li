\documentclass[8pt,table]{bmstu-pr}

\begin{document}

\prtitle{Метод построения поисковых индексов\\в реляционной базе данных\\на основе
глубоких нейронных сетей}{Маслова Марина Дмитриевна}{ИУ7-83Б}{Оленев Антон
Александрович}

\begin{frame}
    \frametitle{Актуальность}

\end{frame}

\begin{frame}
    \frametitle{Цель и задачи}

    {
    \fontsize{22pt}{22pt}\selectfont
    \textbf{Цель:} разработка метода построения поисковых индексов в реляционной
    базе данных на основе глубоких нейронных сетей.

    \vspace{2mm}
    \textbf{Задачи:}
    \begin{itemize}
        \item рассмотреть и сравнить известные методы построения индексов;
        \item привести описание построения индексов с помощью нейронных сетей;
        \item разработать метод построения индексов в реляционной базе
            данных на основе глубоких нейронных сетей;
        \item разработать программное обеспечение, реализующее данный метод;
        \item провести исследование (по времени и памяти) операций поиска и вставки
            с использованием индекса, построенного разработанным методом, при
            различных объемах данных.
    \end{itemize}
    }
\end{frame}

\begin{frame}
    \fontsize{22pt}{22pt}\selectfont
    \frametitle{Сравнение методов построения индексов}

{
\fontsize{20pt}{20pt}\selectfont
\renewcommand{\arraystretch}{1.5}
\begin{longtable}[Hc]{|p{4cm}|p{1.5cm}|p{3cm}|p{4cm}|p{2.5cm}|p{4cm}|}
    %>{\columncolor{green}}
    \hline
    \multicolumn{2}{|c|}{\textbf{Метод}} &
    \multicolumn{1}{c|}{\textbf{B-дерево}} &
    \multicolumn{1}{c|}{\textbf{Хеш-таблица}} &
    \multicolumn{1}{c|}{\parbox{2.5cm}{\centering\textbf{Фильтр Блума}}} &
    \multicolumn{1}{c|}{\parbox{4cm}{\centering\textbf{Обученные индексы}}}\\
    \hline
    \multirow{2}{*}{\parbox{2cm}{Временная\\сложность}} &
    поиска & O(log N) & O(1) / O(N) & O(k) & O(1) / O(N)\\
    \cline{2-6}
    & вставки & O(log N) & O(1) / O(N) & O(k) & (*)\\
    \hline
    \multicolumn{2}{|l|}{Память} & Высокая & Средняя & Низкая & Средняя\\
    \hline
    \multicolumn{2}{|l|}{Поиск в диапазоне} & + & - & - & + \\
    \hline
    \multicolumn{2}{|l|}{Поиск единичного ключа} & + & + & - & + \\
    \hline
    \multicolumn{2}{|l|}{Проверка существования} & + & + & + & + \\
    \hline
\end{longtable}
}

$(*)$ --- вставка в обученный индекс требует переобучения, сложность которого
зависит от архитектуры используемой модели машинного обучения.

\end{frame}

\begin{frame}
    \fontsize{22pt}{22pt}\selectfont
    \frametitle{Постановка задачи}

    \imgfs{idef0-A0}{h!}{1.4}
    Ограничение: ключи --- целые уникальные числа.
\end{frame}


\begin{frame}
    \fontsize{22pt}{22pt}\selectfont
    \frametitle{Структура индекса}

    \begin{minipage}{0.40\textwidth}
        \imgfs{learnedstruct}{h!}{1.6}
    \end{minipage} \hfill
    \begin{minipage}{0.55\textwidth}
        \begin{equation*}
            p = F(K) \cdot N,
        \end{equation*}
        ~\\

        где $p$ --- искомая позиция;

        ~~~~$K$ --- ключ поиска;

        ~~~~$F(K)$ --- функция распределения;

        ~~~~$N$ --- количество ключей.
    \end{minipage}

\end{frame}

\begin{frame}
    \frametitle{Функциональная схема построения индекса}
    \imgfw{idef0-A1}{h!}{20cm}
\end{frame}

\begin{frame}
    \frametitle{Архитектура нейронной сети}
    \imgfw{fcnn2}{h!}{10cm}
    \imgfw{fcnn3}{h!}{10cm}
\end{frame}

\begin{frame}
    \frametitle{Функциональная схема поиска}
    \imgfw{search-A1-big}{h!}{24cm}
\end{frame}

\begin{frame}
    \frametitle{Функциональная схема вставки}
    \imgfw{insert-A1}{h!}{22cm}
\end{frame}


\begin{frame}
    \frametitle{Структура программного обеспечения}
    \imgfw{sw-struct}{h!}{23cm}
\end{frame}

\begin{frame}
    \frametitle{Исследование времени построения}
    %\begin{minipage}{0.45\textwidth}
    %    \imgfs{res-build-distr}{h!}{1}
    %\end{minipage}
    \begin{minipage}{\textwidth}
        \imgfs{res-build}{h!}{1.2}
    \end{minipage}
\end{frame}

\begin{frame}
    \frametitle{Исследование времени поиска (распределения)}
    \begin{minipage}{0.45\textwidth}
        \imgfs{res-error-distrs}{h!}{0.7}
    \end{minipage}
    \begin{minipage}{0.45\textwidth}
        \imgfs{res-search-distrs}{h!}{0.7}
    \end{minipage}

\end{frame}

\begin{frame}
    \frametitle{Исследование времени поиска (модели)}

    \begin{minipage}{0.45\textwidth}
        \imgfs{res-error-2vs3}{h!}{0.7}
    \end{minipage}
    \begin{minipage}{0.45\textwidth}
        \imgfs{res-search-2vs3}{h!}{0.7}
    \end{minipage}

\end{frame}

\begin{frame}
    \frametitle{Исследование времени поиска (этапы)}
    \begin{minipage}{0.45\textwidth}
        \imgfs{res-search-steps}{h!}{0.7}
    \end{minipage}
    \begin{minipage}{0.45\textwidth}
        \imgfs{res-histogram}{h!}{0.7}
    \end{minipage}

\end{frame}

\begin{frame}
    \frametitle{Исследование времени вставки}

    \imgfs{res-insert-sqlite}{h!}{1}
\end{frame}

\begin{frame}
    \frametitle{Исследование памяти, занимаемой индексом}

    \imgfs{res-memory-sqlite}{h!}{1}
\end{frame}

\begin{frame}
    \fontsize{22pt}{22pt}\selectfont
    \frametitle{Заключение}
    В ходе данной работы:
    \begin{itemize}
        \item проанализированы известные методы построения индексов;
        \item приведено описание построения индексов с помощью нейронных сетей;
        \item разработан метод построения индексов в реляционной базе
            данных на основе глубоких нейронных сетей;
        \item разработано программное обеспечение, реализующее данный метод;
        \item провестно исследование (по времени и памяти) операций поиска и
            вставки с использованием индекса, построенного разработанным
            методом, при различных объемах данных.
    \end{itemize}
    ~\\

    Поставленная цель достигнута.
\end{frame}

\begin{frame}
    \fontsize{22pt}{22pt}\selectfont
    \frametitle{Дальнейшее развитие}
    \begin{enumerate}
        \item Оптимизация алгоритма вставки с учетом распределения ключей.
        \item Добавление возможности построения индекса по ключам других типов
            данных.
        \item  Построение многомерных обученных индексов.
    \end{enumerate}
    ~\\
\end{frame}

\end{document}

