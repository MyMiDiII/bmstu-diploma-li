\documentclass{bmstu-pr}

\begin{document}

\prtitle{Метод построения поисковых индексов\\в реляционной базе данных\\на основе
глубоких нейронных сетей}{Маслова Марина Дмитриевна}{ИУ7-83Б}{Оленев Антон
Александрович}

\begin{frame}
    \frametitle{Цель и задачи}

    \textbf{Цель:} цель.

    \textbf{Задачи:}
    \begin{itemize}
        \item задача 1;
        \item задача 2;
        \item задача 3;
        \item задача 4.
    \end{itemize}
\end{frame}

\begin{frame}
    \frametitle{Важный слайд}

    %Индекс --- это некоторая структура, обеспечивающая быстрый поиск записей в
    %базе данных.
    %~\\
    %~\\
    %Индекс:
    %\begin{itemize}
    %    \item определяет соответствие ключа поиска конкретной записи с
    %        положением этой записи;
    %    \item строится в дополнение к существующим данным.
    %\end{itemize}
\end{frame}

\begin{frame}
    \frametitle{Важный слайд}

    %\begin{itemize}
    %    \item кластеризованные и некластеризованные;
    %    \item плотные и разреженные;
    %    \item одноуровневые и многоуровневые.
    %\end{itemize}

    %\hspace*{\fill}%
    %\raisebox{-\height}{\imgs{dense}{0.8}}%
    %\hfill
    %\raisebox{-\height}{\imgs{sparse}{0.8}}%
    %\hfill
    %\raisebox{-\height}{\imgs{multilevel}{0.8}}%
    %\hspace*{\fill}

\end{frame}


\begin{frame}
    \frametitle{Важный слайд}

    %\centering B-деревья
    %\imgfs{bplustreecrop}{h!}{1}
    %\centering B$^+$-деревья
    %\imgfs{btreecrop}{h!}{1}
\end{frame}

\begin{frame}
    \frametitle{Важный слайд}
    %\centering\imgs{blearnedcomp}{1.5}
\end{frame}

\begin{frame}
    \frametitle{Важный слайд}
    %\centering\imgs{hashlearnedhash}{1.5}
\end{frame}

\begin{frame}
    \frametitle{Важный слайд}
    %\imgfs{bloom}{h!}{1.5}
    %\imgfs{learnedBloom}{h!}{1.5}
\end{frame}

\begin{frame}
    \frametitle{Важный слайд}
    %\centering\imgs{rmi}{1.5}
\end{frame}


\begin{frame}
    \frametitle{Важный слайд}

    %\begin{itemize}
    %    \item индексы для поиска в диапазоне;
    %\end{itemize}
\end{frame}

\begin{frame}
    \frametitle{Важный слайд}

\end{frame}

\begin{frame}
    \frametitle{Заключение}
    В ходе данной работы:
    \begin{itemize}
        \item
        \item
        \item
    \end{itemize}
    ~\\

    Поставленная цель достигнута.
\end{frame}

\begin{frame}
    \frametitle{Дальнейшее развитие}
    \begin{itemize}
        \item важное 1;
        \item важное 2;
        \item важное 3;
    \end{itemize}
    ~\\
\end{frame}

\end{document}

