\chapter{\label{impl}Технологическая часть}

\section{Выбор средств программной реализации}

Для реализации метода построения индекса в реляционной базе данных на основе
глубоких нейроных сетей в качестве языка программирования выбран
Python~3.10~\cite{python}, так как предоставляет широкий выбор библиотек для
глубокого обучения и визуализации его результатов.  В качестве библиотеки
глубокого обучения выбран TensorFlow~2.11.0~\cite{tf} и работающий поверх нее
высокоуровневый программный интерфейс Keras~2.11.0~\cite{keras}.  Для работы с
массивами данных выбрана библиотека numpy~\cite{numpy}.

В качестве реляционной системы управления базами данных выбрана
SQLite~\cite{sqlite}, предоставляющая программный интерфейс виртуальных таблиц,
позволяющих релизовать пользовательский поисковый индекс. Виртуальные таблицы
являются одним из видов расширений SQLite, программный интерфейс которых
предоставляется на языке C~\cite{c}, который и выбран в качестве языка
программирования для взаимодействия с реляционной базой данных.

Для обеспечения взаимодействия между компонентом работы с базой данных и
компонентом, непосредственно реализующий индекс, используются библиотеки языка C
\texttt{Python.h} для работы с объектами языка Python и
\texttt{numpy/arrayobject.h}, предоставляющая программный интерфейс для работы с
numpy-массивами, которые являются основным типом данных, через который
происходит взаимодействие модулей.

\section{Реализация программного обеспечения}

\subsection{Форматы входных и выходных данных}

Основой для построения индекса в качестве входных выступают данные из таблицы
реляционной базы данных SQLite. Требованием к таблице является наличие атрибута
целочисленного типа (\texttt{INTEGER}) с уникальными
значениями~(\texttt{UNIQUE}).

Для создания индекса в аргументах соответствующего запроса должен присутствовать
идентификатор столбца, удовлетворяющего требованию описанному выше, и строковое
имя модели индекса, на основе которой он строится (\texttt{FCNN2} --- для модели
нейронной сети с двумя скрытыми слоями, \texttt{FCNN3} --- с тремя).

В качестве входных данных компонента, реализующего индекса, является набор
значений столбца, идентификатор которого был указан в аргументах запроса на
создание индекса, и идентификаторы строк \texttt{ROWID} индексируемой таблицы.

Выходным данных является указатель на объект, представляющий индекс на основе
глубокой нейронной сети, обученный на предоставленных входных данных.

Формат входных и выходных данных операций с индексом (поиска и вставки) описан в
следующем пункте.

\subsection{Поддерживаемые виды запросов}
%{
%\captionsetup{format=hang,justification=raggedright,
%              singlelinecheck=off,width=16cm}
%\listingfile{sql}{venueTable.sql}{}{Создание таблицы мест проведения}{venues}

\subsection{Программный интерфейс виртуальных таблиц}

структура модуля

функция инициализации

создание и удаление, подключение и отключение таблицы

функции курсора numpy массив

реализации

\subsection{Реализация индекса}

\section{Сборка программного обеспечения}

\section{Взаимодействие с программным обеспечением}

\section{Пример работы}

\section{Результаты тестирования}
