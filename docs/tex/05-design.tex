\chapter{\label{design}Конструкторская часть}

\section{Требования и ограничения метода}

Метод построения поисковых индексов в реляционной базе данных на основе глубоких
нейронных сетей (далее – метод построения индексов) должен:

\begin{enumerate}
    \item получать из таблицы реляционной базы данных набор ключей и набор
        соответствующих указателей на записи в индексируемой таблице или иных
        значений, выполняющих роль указателей;
    \item выполнять предварительную обработку полученных наборов, такую, как их
        совместную сортировку по значениям ключей, получение позиций ключей в
        отсортированном виде и нормализацию ключей и позиций;
    \item обучать модель нейронной сети на подготовленных набора ключей и
        позиций;
    \item сохранять параметры обученной модели для каждой таблицы с целью
        возможности выполнять запросы поиска без переобучения;
    \item обеспечивать поиск записи (диапазона записей) таблицы по ключу
        (диапазону ключей) с использованием обученной модели;
    \item обеспечивать корректность операции поиска после вставки/удаления новых
        записей путем переобучения модели;
\end{enumerate}

На разрабатываемый метод накладываются следующие ограничения:

\begin{itemize}
    \item в качестве ключей на вход принимаются целые числа для исключения
        решения дополнительной задачи преобразования входных данных;
    \item ключи во входном наборе уникальны.
\end{itemize}

\section{Особенности метода построения индекса}

\subsection{Общее описание метода построения индекса}

Основные этапы метода построения индекса приведены на функциональной
декомпозиции метода на рисунке~\ref{img:idef0-A1}. 

\imgw{idef0-A1}{h!}{17cm}{Функциональная схема метода построения индекса}

На вход методу подается набор уникальных целочисленных ключей, которые перед
обучением модели глубокой нейронной сети проходят предварительную обработку по
определенным правилам, описанным далее. Отдельным этапом выделено получение
значений функций распределения для каждого ключа, относящееся к предварительной
обработке, но представляющее собой ее ключевой момент. Полученные после первых
двух этапов обработанные ключи и соответствующие значения функций используются
для обучение модели глубокой нейронной сети в качестве признаков и меток
соответственно.

Ключевым моментом метода является представление в отсортированном (по ключам)
виде наборов ключей и набора соответвующих указателей на данные. Именно
отсортированный вид позволяет использовать закономерность распределения ключей
по позициям для обучения модели, предсказывать позиции ключей и уточнять их.

Результатом работы метода является структура данных, предствляющая собой индекс
на основе глубокой нейронной сети и имеющая следующие поля:

\begin{itemize}
    \item отсортированный массив ключей, поданных на вход;
    \item отсортированный по значениям ключей массив указателей
        на данные, соответствующие ключам;
    \item модель обученной глубокой нейронной сети, с помощью которой будет
        предсказываться положение ключа в отсортированном массиве;
    \item средняя и максимальная абсолютные ошибки предсказания позиции ключа,
        для ее уточнения и возврата верного указателя на данные.
\end{itemize}

Краткое описания индекса, являющегося результатом работы метода, как структуры
данных представлено на рисунке~\ref{img:index-struct}.

\imgs{index-struct}{h!}{1}{Индекс как структура данных}

Подробное описание каждого этапа приведено в следующих пунктах данного
подраздела.

\subsection{Предварительная обработка данных}

Разрабатываемый метод построения индекса предполагает предварительную обработку
набора целочисленных ключей, полный алгоритм которой представлен на
рисунке~\ref{img:preprocess}.

\imgw{preprocess}{h!}{5cm}{Схема алгоритма предварительной обработки данных}

Требуется нормализовывать ключи в диапазон $[0, 1]$, поэтому используется метод
минимакс-нормализации, при котором нормализованное значение вычисляется по
формуле:

\begin{equation}
    x_{\text{норм}}= \frac{x - x_{min}}{x_{max} - x_{min}}
\end{equation}

Значение функции распределения $F$ некоторого ключа $K$ зависит от позиции ключа
$P$ и количества индексируемых ключей $N$ и вычисляется по формуле:

\begin{equation}
    F(K) = \frac{P}{N}
\end{equation}

\subsection{Разработка архитектуры глубокой нейронной сети}
…

Получаемая позиция требует уточнения, происходящего за счет получаемого в
результате обучения модели максимального отклонения от истинного расположения.
(Алгоритм уточнения???)

\section{Функциональная схема использования индекса}
Основной операцией, выполняемой с помощью индекса, является поиск,
функциональная схема выполнения которого представлена на
рисунках~\ref{img:search-A0}-\ref{img:search-A1}.

\imgs{search-A0}{h!}{1}{Функциональная схема нулевого уровня поиска}
\imgw{search-A1}{h!}{17cm}{Функциональная схема первого уровня поиска}

Для реализации вставки происходит добавление новых значений ключа и указателя в
существующие массивы, и повторятся алгоритм построения индекса (??? не сначала,
а со значений параметров уже обученной модели ???).

\section{Данные для обучения и тестирования индекса}

Так как в основе индекса на основе глубоких нейронных сетей лежит аппроксимация
функции распределения ключей, работу разработанного метода.

\begin{itemize}
    \item Равномерное
    \item Нормальное
    \item Экспоненциальное
    \item Реальные данные (osm, face, wiki ???)
\end{itemize}

