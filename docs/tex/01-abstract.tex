\begin{essay}{}
    \noindent\mbox{ИНДЕКСЫ,} \mbox{B-ДЕРЕВЬЯ,} \mbox{ХЕШ-ИНДЕКСЫ,}
    \mbox{БИТОВЫЕ} \mbox{ИНДЕКСЫ,} \mbox{ОБУЧЕННЫЕ} \mbox{ИНДЕКСЫ,} \mbox{БАЗЫ}
    \mbox{ДАННЫХ,} \mbox{СИСТЕМЫ} \mbox{УПРАВЛЕНИЯ} \mbox{БАЗАМИ} \mbox{ДАННЫХ}

    Объектом исследования является построение индексов в базах данных.

    Цель работы --- классификация методов построения индексов в базах данных.

    В разделе~\ref{analysis} рассмотрено понятие индекса в базах данных и его
    основные свойства, а также описаны типы индексов.

    В разделе~\ref{methods} проведен обзор методов построения индексов на основе
    B-деревьев, хеш-таблиц и битовых карт, а также соответствующих обученных
    индексов.

    В разделе~\ref{classification} приведены критерии оценки качества описанных
    методов и проведено сравнение по этим критериям.

    В результате работы выявлено, что обученные индексы могут позволить
    уменьшить время поиска~в~$9.8$~раза и вставки~в~$15.7$~раза по сравнению c
    традиционными индексами, сократить число коллизий на~$44.8\%$, а также
    уменьшить размер индексов, решающих задачу проверки существования ключа
    на~$36\%$.

\end{essay}
