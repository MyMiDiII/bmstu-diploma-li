\chapter{\label{research}Исследовательская часть}

\section{Предмет исследования}

Характеристиками, определяющими эффективность индекса являются:

\begin{itemize}
    \item время выполнения основных операций:
\begin{itemize}
    \item построения;
    \item поиска;
    \item вставки (в случае представленного метода являющейся
        комбинацией первых двух операций);
\end{itemize}
    \item память, занимаемая индексом.
\end{itemize}

Так как метод основан на использовании глубокой нейронной сети,
характеристикой также является средняя абсолютная ошибка предсказания позиции
ключей, получаемая в ходе обучения.

Предполагается зависимость описанных характеристик от объема индексируемых
данных, а также от распределения ключей, поэтому исследование проводится на
различном количестве ключей [список значений количества ключей] \textit{для
каждого распределения, описанного в подразделе~\ref{data}: равномерного,
нормального и распределения реальных данных OpenStreetMap}.

\section{Исследование эффективности построения индекса}

На рисунке~\ref{img:res-build} приведен график зависимости времени построения
индекса от количества ключей в индексируемом наборе данных.

\textit{Таблица???}

%\imgw{res-build}{h!}{17cm}{График зависимости времени построения
%индекса от количества ключей}

Вывод, вывод, вывод... Линейная зависимость...

\section{Исследование эффективности поиска}

На время поиска с использованием индекса, построенного с помощью разработанного
метода, должна оказывать влияние абсолютная ошибка предсказания позиции ключа
моделью глубокой нейронной сети, так как она определяет диапазон, в котором
осуществляется уточнение с помощью бинарного поиска.

График зависимости среденей абсолютной ошибки от количества ключей представлен
на рисунке~\ref{img:res-mean-ae}. Нормированное распределение абсолютной ошибки
представлено на рисунке~\ref{img:res-histogram}.

Подвывод об ошибке... Принимает некоторое постоянное значение в процентах к
числу ключей. => диапазон бинарного поиска линейно растет.

На рисунке~\ref{img:res-search} представлен график зависимости времени поиска от
числа ключей.

\textit{ДОБАВИТЬ ГРАФИК ВРЕМЕНИ БИНАРНОГО ПОИСКА???}

Вывод по времени поиска...

\section{Исследование эффективности вставки}

\textit{ПОИСК + ПОСТРОЕНИЕ}

На рисунке~\ref{img:res-search} представлен график зависимости времени вставки
от числа ключей.

\section{Исследование памяти, используемой индексом}

На рисунке~\ref{img:res-memory} представлен график зависимости размера индекса
от количества ключей.

\texttt{модель + размер массива}
