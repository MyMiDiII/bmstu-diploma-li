\chapter{\label{research}Исследовательская часть}

\section{Предмет исследования}

Характеристиками, определяющими эффективность индекса являются:

\begin{itemize}
    \item время выполнения основных операций:
\begin{itemize}[wide=\dimexpr\parindent-2em+\labelsep\relax, leftmargin=* ]
    \item построения;
    \item поиска;
    \item вставки (являющейся
        комбинацией первых двух операций);
\end{itemize}
    \item память, занимаемая индексом.
\end{itemize}

Так как метод основан на использовании глубокой нейронной сети,
характеристикой также является средняя абсолютная ошибка предсказания позиции
ключей, получаемая в ходе обучения.

Предполагается зависимость описанных характеристик от объема индексируемых
данных, а также от распределения ключей, поэтому исследование проводится на
различном количестве ключей для распределений, описанных в
подразделе~\ref{data}: равномерного, нормального и распределения реальных данных
OpenStreetMap. Также проводится сравнение времени поиска при моделях с различным
числом скрытых слоев, описанных в подразделе~\ref{dnn}, и с реализацией
классического индекса в SQLite.

\section{Зависимость времени построения индекса от количества ключей}

На рисунке~\ref{img:res-build} приведен график зависимости времени построения
индекса от количества ключей в индексируемом наборе OpenStreetMap при различных
количествах скрытых слоев в модели в сравнении с классическим индексом SQLite.

Из приведенных графиков можно сделать вывод, что добавление дополнительного слоя
в модель увеличивает время обучения, а следовательно и построения. По сравнению
с индексом на основе модели с двумя скрытыми слоями индекс на основе модели с
тремя увеличивает время построения на~10\%. Время построения же классического
индекса в SQLite в 25~раз ниже построения реализованного индекса.

По полученным данным исследования строится линейная регрессионная модель, откуда
вытекает линейная зависимость времени построения индекса от количества ключей в
наборе, что объясняется достижением необходимой точности модели за
\mbox{1-2~эпохи} обучения при каждом размере данных, за которые происходит
проход по всем значениям ключей. Аналогичная зависимость наблюдается и у
классического индекса.

\imgw{res-build}{h!}{17cm}{График зависимости времени построения
индекса от количества ключей (модели, OpenStreetMap)}


\section{Исследование времени поиска}

На время поиска с использованием индекса, построенного с помощью разработанного
метода, должна оказывать влияние абсолютная ошибка предсказания позиции ключа
моделью глубокой нейронной сети, так как она определяет диапазон, в котором
осуществляется уточнение с помощью бинарного поиска, поэтому в дополнение к
исследованию зависимости времени поиска от количества ключей было также
проведено исследование зависимости средней абсолютной ошибки от количества
ключей.

На рисунка~\ref{img:res-error-distrs} и \ref{img:res-search-distrs} приведены
графики зависимостей средней абсолютно ошибки и времени поиска от количества
ключей при различных распределениях с использованием модели с двумя скрытыми
слоями.

\imgw{res-error-distrs}{h!}{17cm}{График зависимости средней абсолютной ошибки 
от количества ключей (распределения, 2 скрытых слоя)}

\imgw{res-search-distrs}{h!}{16cm}{График зависимости времени поиска 
от количества ключей (распределения, 2 скрытых слоя)}

По графикам на данных, распределенных по равномерному закону, функция
распределения которых является линейной, имеет наименьшее значение абсолютной
ошибки и, как следствие, меньшее время поиска. Наибольшая абсолютная ошибка и
время поиска наблюдаются на реальных данных, так как график их функция
распределения имеет более сложный вид нежели классические распределения. Однако
при росте ошибки на~0.2\% относительно равномерного распределения, время поиска
на реальных данных увеличивается в среднем на~6\%.

При этом при увеличении числа ключей отношение средней абсолютной ошибке к
количеству ключей стремится к некотором постоянному значению, то есть диапазон
бинарного поиска будет составлять некоторую постоянную часть от числа ключей, то
есть будет линейно расти.

На рисунка~\ref{img:res-error-2vs3} и \ref{img:res-search-2vs3} приведены
графики зависимостей средней абсолютно ошибки и времени поиска от количества
ключей при реальных данных с использованием моделей с двумя и тремя слоями.

\imgw{res-error-2vs3}{h!}{16cm}{График зависимости средней абсолютной ошибки 
от количества ключей (модели, OpenStreetMap)}

\imgw{res-search-2vs3}{h!}{17cm}{График зависимости времени поиска 
от количества ключей (модели, OpenStreetMap)}

Таким образом, добавлении третьего слоя оказалось неоправданным, достигаемое
уменьшение средней абсолютной ошибки на~0.06\% и уменьшение времени бинарного
поиска превысилось временем, затрачиваемым на подсчет дополнительных
коэффициентов третьего слоя при предсказании.

На рисунке~\ref{img:res-search-steps} представлен график зависимости времени
поиска и его этапов (предсказания и уточнения) от числа ключей на реальных
данных с использованием модели с двумя скрытыми слоями.

\imgw{res-search-steps}{h!}{15cm}{График зависимости времени поиска,
предсказания и уточнения от количества ключей (OpenStreetMap, 2 скрытых слоя)}

Таким образом, предсказание выполняется за постоянное время, которое больше
времени выполнения бинарного поиска с временной сложностью $O(log N)$. Таким
образом, сложность поиска с использованием индекса, построенного разработанным
методом, $O(log N)$. При этом коэффициент при $N$ равен величине отношения
средней абсолютной ошибки к числу ключей, которая составляет 0.2\%.

Для оценки худшего случая построим нормированное распределение абсолютной
ошибки, которое представлено на рисунке~\ref{img:res-histogram}.

\imgw{res-histogram}{h!}{17cm}{Распределение абсолютной ошибки предсказания
позиции}

Из графика можно сделать вывод, что максимальная абсолютная ошибка составляет
0.6\% от числа ключей, то есть в худшем случае бинарный поиск будет осуществляться
в диапазоне $0.006N$.

На рисунке~\ref{img:res-search-sqlite} представлен график зависимости времени
поиска  от числа ключей на реальных
данных с использованием модели с двумя скрытыми слоями в сравнении с временем
поиска без индекса в SQLite и с классическим индексом SQLite.

\imgw{res-search-sqlite}{h!}{17cm}{Зависимость времени поиска от числа ключей с
помощью разработанного индекса, без индекса в SQLite и с классическим индексом
SQLite}

Таким образом, время поиска с использованием обоих индексов в среднем 17500 раз
меньше, чем без индекса. При этом на взятых количествах ключей время поиска с
индексом SQLite меньше, чем время поиска с использованием разработанного
индекса, однако время поиска с использованием классического индекса SQLite
растет быстрее времени поиска с использованием разработанного индекса при
одинаковом проценте увеличения числа ключей. Так, при увеличении числа ключей в
2~раза время поиска SQLite увеличивается в~1.25~раза, а время поиска
разработанного индекса в~1.03~раза.

\section{Исследование времени вставки}

На рисунке~\ref{img:res-insert} представлен график зависимости времени вставки
от числа ключей с использованием разработанного индекса и классического индекса
SQLite.

\imgw{res-insert}{h!}{17cm}{График зависимости времени вставки от количества
ключей (реальные данные, 2 скрытых слоя)}

Вставка представляет собой комбинацию поиска и построения индекса, поэтому
зависимость для разработанного индекса является линейкой, как у построения.  При
этом время вставки в разработанный индекс в 6.7~раза больше классического при
максимальном количестве ключей.

\section{Исследование памяти, используемой индексом}

%На рисунке~\ref{img:res-memory} представлен график зависимости размера индекса
%от количества ключей.
%
%\imgw{res-memory}{h!}{17cm}{График зависимости дополнительной памяти, занимаемой
%индесом, от количества ключей}

На рисунке~\ref{img:res-memory} представлен график зависимости размера индекса
от количества ключей.

\imgw{res-memory}{h!}{17cm}{График зависимости дополнительной памяти,
занимаемой индесом, от количества ключей}

Индекс, построенный разработанным методом, занимает место только под хранение
модели, имеющей постоянный размер, который не зависит от числа ключей, и под два
массива, память под который увеличивается линейно. То есть, если учитывать
только память, выделяемую на поддержку структуры индекса, разработанный индекс
имеет постоянный размер, соответствующий размеру модели нейронной сети, в отличие
от традиционных индексов, в которых выделяется память под указатели для
поддержки структуры, число которых зависит от числа ключей. 

\section{Результаты исследования}

В ходе исследования были сделаны следующие выводы:
\begin{itemize}
    \item временная сложность построения $O(N)$;
    \item средняя абсолютная ошибка предсказания модели стремится к некоторому
        постоянному значению по отношению к общему числу ключей, что дает
        линейную зависимость размера диапазона для бинарного поиска от числа
        ключей, причем с малым коэффициентом, равным~$0.002$ --- в среднем 
        случае, и~$0.006$ --- в худшем для реальных данных;
    \item разработанный индекс имеет временную сложность поиска $O(\log N)$;
    \item временная сложность вставки составляет $O(N)$;
    \item разработанный индекс затрачивает на поддержание своей структуры не
        зависящее от числа данных количество памяти;
    \item с учетом хранимых ключей и указателей, память, затрачиваемая индексом,
        линейно растет при увеличении числа индексируемых ключей;
    \item увеличение точность путем добавления дополнительных слоев в нейронную
        сеть не оправданно из-за роста затрат на вычисления дополнительных
        коэффициентов третьего слоя;
    \item индекс быстрее работает на классических распределениях.
\end{itemize}

