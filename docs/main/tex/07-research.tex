\chapter{\label{research}Исследовательская часть}

\section{Предмет исследования}

Характеристиками, определяющими эффективность индекса являются:

\begin{itemize}
    \item время выполнения основных операций:
\begin{itemize}[wide=\dimexpr\parindent-2em+\labelsep\relax, leftmargin=* ]
    \item построения;
    \item поиска;
    \item вставки (являющейся
        комбинацией первых двух операций);
\end{itemize}
    \item память, занимаемая индексом.
\end{itemize}

Так как метод основан на использовании глубокой нейронной сети,
характеристикой также является средняя абсолютная ошибка предсказания позиции
ключей, получаемая в ходе обучения.

Предполагается зависимость описанных характеристик от объема индексируемых
данных, а также от распределения ключей, поэтому исследование проводится на
различном количестве ключей для распределений, описанных в
подразделе~\ref{data}: равномерного, нормального и распределения реальных данных
OpenStreetMap. Также проводится сравнение времени поиска при моделях с различным
числом скрытых слоев, описанных в подразделе~\ref{dnn}.

\section{Исследование времени построения индекса}

На рисунке~\ref{img:res-build-distr} приведен график зависимости времени
построения индекса от количества ключей в индексируемом наборе данных при
различных распределениях с использованием модели с двумя скрытыми слоями.

\imgw{res-build-distr}{h!}{17cm}{График зависимости времени построения
индекса от количества ключей (распределения, 2 скрытых слоя)}

На рисунке~\ref{img:res-build} приведен график зависимости времени построения
индекса от количества ключей в индексируемом наборе реальных данных при
различных количествах скрытых слоев в модели.

\imgw{res-build}{h!}{17cm}{График зависимости времени построения
индекса от количества ключей (модели, реальные данные)}

Из приведенных графиков можно сделать вывод, что распределение данных не
оказывает влияние на время построения индекса в силу необходимости прохода по
всему набору данных при обучении. При этом добавление дополнительного слоя в
модель увеличивает время обучения, а следовательно и построения. По сравнению с
индексом на основе модели с двумя скрытыми слоями индекс на основе модели с
тремя увеличивает время построения на~10\%.

Основным выводом из приведенных графиков является наблюдаемая линейная
зависимость времени построения индекса от количества ключей в наборе, что
объясняется достижением необходимой точности модели за \mbox{1-2~эпохи} обучения
при каждом размере данных, за которые проиходит проход по всем значениям ключей.

\section{Исследование времени поиска}

На время поиска с использованием индекса, построенного с помощью разработанного
метода, должна оказывать влияние абсолютная ошибка предсказания позиции ключа
моделью глубокой нейронной сети, так как она определяет диапазон, в котором
осуществляется уточнение с помощью бинарного поиска.

График зависимости среденей абсолютной ошибки от количества ключей представлен
на рисунке~\ref{img:res-mean-ae}. Нормированное распределение абсолютной ошибки
представлено на рисунке~\ref{img:res-histogram}.

Подвывод об ошибке... Принимает некоторое постоянное значение в процентах к
числу ключей. => диапазон бинарного поиска линейно растет.

На рисунке~\ref{img:res-search} представлен график зависимости времени поиска от
числа ключей.

\textit{ДОБАВИТЬ ГРАФИК ВРЕМЕНИ БИНАРНОГО ПОИСКА???}

Вывод по времени поиска...

\section{Исследование времени вставки}

\textit{ПОИСК + ПОСТРОЕНИЕ}

На рисунке~\ref{img:res-search} представлен график зависимости времени вставки
от числа ключей.

\section{Исследование памяти, используемой индексом}

На рисунке~\ref{img:res-memory} представлен график зависимости размера индекса
от количества ключей.

\texttt{модель + размер массива}
