\begin{essay}{}
    \noindent\mbox{ИНДЕКСЫ,}  \mbox{ОБУЧЕННЫЕ} \mbox{ИНДЕКСЫ,} \mbox{ГЛУБОКИЕ}
    \mbox{НЕЙРОННЫЕ} \mbox{СЕТИ}, \mbox{РЕЛЯЦИОННАЯ} \mbox{БАЗА} \mbox{ДАННЫХ}

    В данной работе представлена разработка метода построения поисковых индексов
    в реляционной базе данных на основе глубоких нейронных сетей.

    В разделе~\ref{analysis} представлено описание построения индексов на основе
    базовых структур и применение методов машинного обучения к построению
    индексов. Проведено сравнение индексов на основе B-деревьев, хеш-таблиц и
    битовых карт с индексами на основе моделей машинного обучения. Описаны
    особенности индексов в реляционных базах данных и применение нейронных сетей
    к построению индексов.

    В разделе~\ref{design} спроектирован метод построения поисковых индексов в
    реляционной базе данных на основе глубоких нейронных сетей. Представлены
    схемы алгоритмов этапов метода и случаев его применения: поиска и вставки.

    В разделе~\ref{impl} спроектировано, разработано и протестировано
    программное обеспечение, реализующее метод.

    В разделе~\ref{research} проведено исследование времени выполнения операций
    построения, поиска и вставки от количества ключей с использованием индекса,
    построенного разработанным методом, и классического индекса.
\end{essay}
