\chapter{\label{impl}Технологическая часть}

\section{Выбор средств программной реализации}

Для реализации метода построения индекса в реляционной базе данных на основе
глубоких нейроных сетей в качестве языка программирования выбран
Python~3.10~\cite{python}, так как предоставляет широкий выбор библиотек для
глубокого обучения и визуализации его результатов.  В качестве библиотеки
глубокого обучения выбран TensorFlow~2.11.0~\cite{tf} и работающий поверх нее
высокоуровневый программный интерфейс Keras~2.11.0~\cite{keras}.  Для работы с
массивами данных выбрана библиотека numpy~\cite{numpy}.

В качестве реляционной системы управления базами данных выбрана
SQLite~\cite{sqlite}, предоставляющая программный интерфейс виртуальных таблиц,
позволяющих релизовать пользовательский поисковый индекс. Виртуальные таблицы
являются одним из видов расширений SQLite, программный интерфейс которых
предоставляется на языке C~\cite{c}, который и выбран в качестве языка
программирования для взаимодействия с реляционной базой данных.

Для обеспечения взаимодействия между компонентом работы с базой данных и
компонентом, непосредственно реализующий индекс, используются библиотеки языка C
\texttt{Python.h} для работы с объектами языка Python и
\texttt{numpy/arrayobject.h}, предоставляющая программный интерфейс для работы с
numpy-массивами, которые являются основным типом данных, через который
происходит взаимодействие модулей.

\section{Реализация программного обеспечения}

\subsection{Форматы входных и выходных данных}

Основой для построения индекса в качестве входных выступают данные из таблицы
реляционной базы данных SQLite. Требованием к таблице является наличие атрибута
целочисленного типа (\texttt{INTEGER}) с уникальными
значениями~(\texttt{UNIQUE}).

Для создания индекса в аргументах соответствующего запроса должен присутствовать
идентификатор столбца, удовлетворяющего требованию описанному выше, и строковое
имя модели индекса, на основе которой он строится (\texttt{FCNN2} --- для модели
с двумя скрытыми слоями, \texttt{FCNN3} --- с тремя).

В качестве входных данных компонента, реализующего индекса, является набор
значений столбца, идентификатор которого был указан в аргументах запроса на
создание индекса, и идентификаторы строк \texttt{ROWID} индексируемой таблицы.

Выходным данных является указатель на объект, представляющий индекс на основе
глубокой нейронной сети, обученный на предоставленных входных данных.

Формат входных и выходных данных операций с индексом (поиска и вставки) описан в
следующем пункте.

\subsection{Поддерживаемые виды запросов}

Разработанное программное обеспечения предоставляет возможность работы только с
запросами фильтрации, представленными оператором \texttt{WHERE} следующими со
следующими условиями:

\begin{itemize}
    \item \texttt{column operator value},

        где \texttt{column} --- имя проиндексированного столбца,

        \texttt{operator} --- одна из операций сравнения: \texttt{=, <, >, <=, >=},

        \texttt{value} --- некоторое целочисленное значение.

    \item \texttt{column BETWEEN value1 AND value2},
        
        где \texttt{column} --- имя проиндексированного столбца,

        \texttt{value1, value2} --- целочисленные значения, представляющие
        нижнюю и верхнюю границы диапазона.
\end{itemize}

Выходным значением из модуля на языке Python, реализующего индекс, по данным
запросам является numpy-массив с соответствующими запросу значениями ROWID, а
результатом работы программного обеспечения --- набор записей таблицы с ROWID из
представленного массива.

Для вставки поддерживается стандартный запрос \texttt{INSERT}, результатом
которого является индекс с переобученной на новых данных моделью. Запросы
удаления и изменения не поддерживаются, так как являются вторичными для оценки
работы метода.

\subsection{Программный интерфейс виртуальных таблиц}

Виртуальные таблицы SQLite~\cite{vtable} --- это объект базы данных,
представляющий с точки зрения инструкций SQL обычную таблицу или представление,
но обрабатывающий запросы посредством вызова функций программного интерфейса,
которые реализуются пользователем.

Виртальные таблицы являются расширением SQLite, регистрация которых происходит с
использованием макросов и фукнции инициализации, представленных на
линстинге~\ref{lst:init}.

{
\captionsetup{format=hang,justification=raggedright,
              singlelinecheck=off,width=16cm}
\listingfile{c}{init.c}{}{Инициализация расширения}{init}
}

При инициализация расширения вирутальной таблицы должен быть зарегистрирован
модуль, описывающийся структурой \texttt{sqlite3\_module}, с помощью функции
регистрации \texttt{sqlite3\_create\_module}, подробное описание которых
представлено на листинге~\ref{lst:module}

{
\captionsetup{format=hang,justification=raggedright,
              singlelinecheck=off,width=16cm}
\listingfile{c}{module.c}{}{Структура и функции для регистрации модуля
виртуальной таблицы}{module}
}


Методы, сигнатуры которых представлены на листинге~\ref{lst:module}, можно
разделить на две группы.

\begin{itemize}
    \item Методы для взаимодействие с таблицей, как с некоторым объектом, к
        которым относятся:
        \begin{itemize}
            \item \texttt{xCreate} --- создание виртуальной таблицы, при
                выполнении соответствующего запроса, представленного на
                листинге~\ref{lst:create-query};

            {
            \captionsetup{format=hang,justification=raggedright,
                          singlelinecheck=off,width=16cm}
            \listingfile{sql}{create.sql}{}{Запрос на создание виртуальной
              таблицы}{create-query}
            }

            \item \texttt{xConnect} --- подключение к виртуальной таблице,
                вызывющийся при выполнении любого запроса к таблице, который
                является первым при повторном подключении к базе данных;
            \item \texttt{xDestroy} --- удаление вирутальной таблицы при
                выполнеивыполнении запроса, представленного на
                листинге~\ref{lst:drop-query};

            {
            \captionsetup{format=hang,justification=raggedright,
                          singlelinecheck=off,width=16cm}
            \listingfile{sql}{drop.sql}{}{Запрос на удаление виртуальной
              таблицы}{drop-query}
            }

            \item \texttt{xDisconnect} --- удаление поключения к виртуальной
                таблице.
        \end{itemize}

    Данные методы работают со структурой \texttt{sqlite3\_vtab}, представленной
    на листинге~\ref{lst:vtab}. Для реализации нужных функциональностей
    указатель на данную структуру включается в пользовательскую, с которой уже
    работают представленные методы посредством преобразования типов. Это дает
    возможность передавать между методами вирутальной таблицы нужные данные.

    {
    \captionsetup{format=hang,justification=raggedright,
                  singlelinecheck=off,width=16cm}
    \listingfile{c}{vtab.c}{}{Структура вирутальной таблицы}{vtab}
    }

    \item Методы прохода по записям таблицы, использующие для этого структуру
        курсора \texttt{sqlite3\_vtab\_cursor}, представленую на
        листинге~\ref{lst:cursor}, над которой также реализуют обертку для
        хранения необходимых для обработки переменных.

        {
        \captionsetup{format=hang,justification=raggedright,
                      singlelinecheck=off,width=16cm}
        \listingfile{c}{cursor.c}{}{Структура курсора}{cursor}
        }

        Данная группа представлена методом обработки вставки, удаления и
        изменения записи (последний) и методами для прохода по записям таблицы
        при поиске:
        \begin{itemize}
            \item \texttt{xOpen} --- создание и инициализации структуры курсора;
            \item \texttt{xBestIndex} --- получение параметров фильтрации и
                выбор лучшего индекса для обработки запроса;
            \item \texttt{xFilter} --- получение соответствующих параметрам
                фильтрации записей, установка курсора на первую из них;
            \item \texttt{xEof} --- проверка окончания списка выбранных записей;
            \item \texttt{xNext} --- переход к следующей записи;
            \item \texttt{xColumn} --- обработка столбца записи;
            \item \texttt{xClose} --- удаление структуры курсора;
            \item \texttt{xUpdate} --- реализация запросов вставки, удаления и
                изменения.
        \end{itemize}
\end{itemize}

Для реализации метода построения индекса используются оберточные структуры для
виртуальной таблицы и курсора, представленные на
листинге~\ref{lst:tab-cur-structs}.

{
\captionsetup{format=hang,justification=raggedright,
              singlelinecheck=off,width=16cm}
\listingfile{c}{tabcurstructs.c}{}{Пользовательские структуры виртуальной
таблицы и курсора}{tab-cur-structs}
}

На листинге~\ref{lst:lindexCreate} представлена реализация метода создания
индекса, реализованного в качестве \texttt{xCreate}. Код инициализации и запуска
обучение индекса в Python через программиный интерфейс приведен на
листинге~\ref{lst:initIndex}.

{
\captionsetup{format=hang,justification=raggedright,
              singlelinecheck=off,width=16cm}
\listingfile{c}{lindexCreate.c}{}{Создание индекса}{lindexCreate}
}

{
\captionsetup{format=hang,justification=raggedright,
              singlelinecheck=off,width=16cm}
\listingfile{c}{initIndex.c}{}{Инициализация индекса}{initIndex}
}

Обработка параметров фильтрации приведена на листинге~\ref{lst:xBestIndex}.
Получение массива подходящих строк и проход для их вывода приведены на
листингах~\ref{lst:xFilter}, \ref{lst:xSearch} соответственно.

{
\captionsetup{format=hang,justification=raggedright,
              singlelinecheck=off,width=16cm}
\listingfile{c}{xBestIndex.c}{}{Обработка параметров фильтрации
запроса}{xBestIndex}
}

{
\captionsetup{format=hang,justification=raggedright,
              singlelinecheck=off,width=16cm}
\listingfile{c}{xFilter.c}{}{Выбор строк, удовлетворяющих фильтру}{xFilter}
}

{
\captionsetup{format=hang,justification=raggedright,
              singlelinecheck=off,width=16cm}
\listingfile{c}{xSearch.c}{}{Реализация работы курсора}{xSearch}
}

\subsection{Реализация индекса}

...

\section{Сборка программного обеспечения}

При сборке расширения SQLite компиляция и линковка происходит с флагами,
обычно используемыми при сборке динамический библиотек: флаг \texttt{-fPIC} при
компиляции в объектные файлы для создание позиционно-независимого кода и флаг
\texttt{-shared} для получения файла динамической библиотеки. Дополнительными
флагами при линковке являются флаги подключения библиотек \texttt{-lsqlite3} и
\texttt{-lpython3.10}. Также при компиляции требуется указание путей к
заголовочным файлам \texttt{Python.h} и \texttt{numpy/arrayobject.h}. Их
автоматическое получения, а также ключевые моменты сборки приведены на
листинге~\ref{lst:makefile}.

{
\captionsetup{format=hang,justification=raggedright,
              singlelinecheck=off,width=16cm}
\listingfile{make}{makefile}{}{Ключевые моменты сборки программного обеспечения}{makefile}
}

Для работы компонента индекса, реализованного на Python, требуется установка
зависимостей из уже сформированного файла \texttt{requirements.txt} путем
выполения команды, представленной на листинге~\ref{lst:pybuild}. Также для
штатной работы программного обеспечения требуется прописать путь к модулям,
реализованным на языке Python, что приведено на том же листинге.

{
\captionsetup{format=hang,justification=raggedright,
              singlelinecheck=off,width=16cm}
\listingfile{bash}{pybuild.sh}{}{Подготовка для работы модулей Python}{pybuild}
}

\section{Взаимодействие с программным обеспечением}

Взаимодействие с программным обеспечением проиходит через командную строку
sqlite3. Пример работы представлен на листинге~\ref{lsl:example}.

{
\captionsetup{format=hang,justification=raggedright,
              singlelinecheck=off,width=16cm}
\listingfile{bash}{example.sh}{}{Пример работы программного обеспечения}{example}
}

\section{Результаты тестирования}

???
