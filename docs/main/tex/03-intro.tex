\chapter*{ВВЕДЕНИЕ}
\addcontentsline{toc}{chapter}{ВВЕДЕНИЕ}

На протяжении последнего десятилетия происходит автоматизация все большего числа
сфер человеческой деятельности~\cite{koptenok}, что приводит к росту числа
данных. Так, по исследованию компании IDC~(International Data Corporation),
изучающей мировой рынок информационных технологий и тенденций развития
технологий, объем данных к 2025~году составит около 175~зеттабайт, в то время
как на год исследования их объем составлял 33~зеттабайта~\cite{idc}.

Для хранения накопленных данных используются базы данных~(БД), доступ к ним
обеспечивается системами управления базами данных~(СУБД), обрабатывающими
запросы на поиск, вставку, удаление или обновление. При больших объемах
информации необходимы методы для уменьшения времени обработки запросов, одним из
которых является построение индексов~\cite{bits}.

Базовые методы построения индексов основаны на таких структурах, как деревья
поиска, хеш-таблицы и битовые карты~\cite{dama}. Однако с ростом объема данных
требуется модификация существующих или разработка новых структур для уменьшения
времени поиска и затрат на перестроение индекса при изменении данных, а также
сокращения дополнительно используемой памяти. Одним из решений являются
обученные индексы~(\textit{learned~indexes})~\cite{main}, которые представляют
совокупность способов построения, основанных на использовании различных моделей
машинного обучения от линейной регрессии до нейронных сетей, позволяющих
учитывать, в отличие от индексов на базовых структурах, распределение
индексируемых данных. На этой идее строится данная работа.

Целью данной работы является разработка метода построения поисковых индексов в
реляционной базе данных на основе глубоких нейронных сетей.

Для достижения поставленной цели требуется решить следующие задачи:
\begin{itemize}
    \item рассмотреть и сравнить известные методы построения индексов;
    \item привести описание построения индексов с помощью нейронных сетей;
    \item разработать метод построения индексов в реляционной базе
        данных на основе глубоких нейронных сетей;
    \item разработать программное обеспечение, реализующее данный метод;
    \item провести исследование (по времени и памяти) операций поиска и вставки
        с использованием индекса, построенного разработанным методом, при
        различных объемах данных.
\end{itemize}
