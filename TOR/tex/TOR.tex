\documentclass{bmstu}

\addbibresource{biblio.bib}

\begin{document}

\section*{Фамилия и инициалы}

Маслова М. Д.

\section*{Фамилия имя отчество}

Маслова Марина Дмитриевна

\section*{Группа}

ИУ7-73Б

\section*{Тема ВКР}

Метод построения поисковых индексов в реляционной базе данных на основе глубоких
нейронных сетей

\section*{Аналитический раздел}

Провести анализ предметной области. Рассмотреть известные методы построения
поисковых индексов в реляционных базах данных\textit{[, привести результаты
сравнительного анализа}. \textit{Рассмотреть построение индексов с
использованием нейронных сетей.]} Описать формальную постановку задачи в виде
IDF0-диаграммы.

\section*{Конструкторский раздел}

Разработать метод построения индексов в реляционной базе данных на основе
глубоких нейронных сетей. \textit{Изложить особенности предлагаемого метода.}
Сформулировать и описать основные алгоритмы в виде схем. Описать структуры
данных, используемые в алгоритмах.

\section*{Технологический раздел}

Обосновать выбор средств программной реализации. Описать формат входных и
выходных данных. Разработать программное обеспечение, реализующее представленный
метод. Описать взаимодействие пользователя с программным обеспечением.

\section*{Исследовательский раздел}

Провести исследование эффективности разработанного метода построения индексов
при различных параметрах системы (\textit{например, число слоев}). На основе
полученных значений провести сравнение с существующими аналогами.

\section*{Фамилия и инициалы руководителя}

Оленев А. А.

\section*{Фамилия имя отчество в родительном падеже}

Масловой Марины Дмитриевны

\end{document}
