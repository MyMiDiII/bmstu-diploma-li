\documentclass{bmstu}

\addbibresource{biblio.bib}

\begin{document}

\section*{Фамилия и инициалы}

Маслова М. Д.

\section*{Фамилия имя отчество}

Маслова Марина Дмитриевна

\section*{Группа}

ИУ7-73Б

\section*{Тема ВКР}

Метод построения поисковых индексов в реляционной базе данных на основе глубоких
нейронных сетей

\section*{Аналитический раздел}

Провести анализ предметной области. Рассмотреть известные методы построения
поисковых индексов в реляционных базах данных, привести результаты
сравнительного анализа. Рассмотреть построение индексов с использованием
нейронных сетей. Описать формальную постановку задачи в виде IDF0-диаграммы.

\section*{Конструкторский раздел}

Разработать метод построения индексов в реляционной базе данных на основе
глубоких нейронных сетей. Изложить особенности предлагаемого метода.
Сформулировать и описать основные алгоритмы в виде схем. Разработать структуру
нейронной сети. Описать структуры данных, используемые в алгоритмах.

\section*{Технологический раздел}

Обосновать выбор средств программной реализации. Описать формат входных и
выходных данных. Разработать и протестировать программное обеспечение,
реализующее представленный метод. Описать взаимодействие
пользователя с программным обеспечением.

\section*{Исследовательский раздел}

Провести исследование временной эффективности поиска и вставки с использованием
индекса, построенного разработанным методом, при различных объемах данных. На
основе полученных значений провести сравнение с существующими аналогами.

\subsection*{Вопросы}
\begin{itemize}
    \item нужно ли добавить про эффективность по используемой памяти?
    \item вставка, скорее всего, выродится в переобучение модели, это нормально?
\end{itemize}

\section*{Фамилия и инициалы руководителя}

Оленев А. А.

\section*{Фамилия имя отчество в родительном падеже}

Масловой Марины Дмитриевны

\end{document}
